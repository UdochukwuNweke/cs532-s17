%%%%%%%%%%%%%%%%%%%%%%%%%%%%%%%%%%%%%%%%%
% Programming/Coding Assignment
% LaTeX Template
%
% This template has been downloaded from:
% http://www.latextemplates.com
%
% Original author:
% Ted Pavlic (http://www.tedpavlic.com)
%
% Note:
% The \lipsum[#] commands throughout this template generate dummy text
% to fill the template out. These commands should all be removed when 
% writing assignment content.
%
% This template uses a Perl script as an example snippet of code, most other
% languages are also usable. Configure them in the "CODE INCLUSION 
% CONFIGURATION" section.
%
%%%%%%%%%%%%%%%%%%%%%%%%%%%%%%%%%%%%%%%%%

%----------------------------------------------------------------------------------------
%	PACKAGES AND OTHER DOCUMENT CONFIGURATIONS
%----------------------------------------------------------------------------------------

\documentclass{article}

\usepackage{fancyhdr} % Required for custom headers
\usepackage{lastpage} % Required to determine the last page for the footer
\usepackage{extramarks} % Required for headers and footers
\usepackage[usenames,dvipsnames]{color} % Required for custom colors
\usepackage{graphicx} % Required to insert images
\usepackage{listings} % Required for insertion of code
\usepackage{courier} % Required for the courier font
\usepackage{lipsum} % Used for inserting dummy 'Lorem ipsum' text into the template

% Margins
\topmargin=-0.45in
\evensidemargin=0in
\oddsidemargin=0in
\textwidth=6.5in
\textheight=9.0in
\headsep=0.25in

\linespread{1.1} % Line spacing

% Set up the header and footer
\pagestyle{fancy}
\lhead{\hmwkAuthorName} % Top left header
\chead{\hmwkClass\ (\hmwkClassInstructor\ \hmwkClassTime): \hmwkTitle} % Top center head
\rhead{\firstxmark} % Top right header
\lfoot{\lastxmark} % Bottom left footer
\cfoot{} % Bottom center footer
\rfoot{Page\ \thepage\ of\ \protect\pageref{LastPage}} % Bottom right footer
\renewcommand\headrulewidth{0.4pt} % Size of the header rule
\renewcommand\footrulewidth{0.4pt} % Size of the footer rule

\setlength\parindent{0pt} % Removes all indentation from paragraphs

%----------------------------------------------------------------------------------------
%	CODE INCLUSION CONFIGURATION
%----------------------------------------------------------------------------------------

\definecolor{MyDarkGreen}{rgb}{0.0,0.4,0.0} % This is the color used for comments
\lstloadlanguages{Perl} % Load Perl syntax for listings, for a list of other languages supported see: ftp://ftp.tex.ac.uk/tex-archive/macros/latex/contrib/listings/listings.pdf
\lstset{language=Perl, % Use Perl in this example
        frame=single, % Single frame around code
        basicstyle=\small\ttfamily, % Use small true type font
        keywordstyle=[1]\color{Blue}\bf, % Perl functions bold and blue
        keywordstyle=[2]\color{Purple}, % Perl function arguments purple
        keywordstyle=[3]\color{Blue}\underbar, % Custom functions underlined and blue
        identifierstyle=, % Nothing special about identifiers                                         
        commentstyle=\usefont{T1}{pcr}{m}{sl}\color{MyDarkGreen}\small, % Comments small dark green courier font
        stringstyle=\color{Purple}, % Strings are purple
        showstringspaces=false, % Don't put marks in string spaces
        tabsize=5, % 5 spaces per tab
        %
        % Put standard Perl functions not included in the default language here
        morekeywords={rand},
        %
        % Put Perl function parameters here
        morekeywords=[2]{on, off, interp},
        %
        % Put user defined functions here
        morekeywords=[3]{test},
       	%
        morecomment=[l][\color{Blue}]{...}, % Line continuation (...) like blue comment
        numbers=left, % Line numbers on left
        firstnumber=1, % Line numbers start with line 1
        numberstyle=\tiny\color{Blue}, % Line numbers are blue and small
        stepnumber=5 % Line numbers go in steps of 5
}

% Creates a new command to include a perl script, the first parameter is the filename of the script (without .pl), the second parameter is the caption
\newcommand{\perlscript}[2]{
\begin{itemize}
\item[]\lstinputlisting[caption=#2,label=#1]{#1.pl}
\end{itemize}
}

%----------------------------------------------------------------------------------------
%	DOCUMENT STRUCTURE COMMANDS
%	Skip this unless you know what you're doing
%----------------------------------------------------------------------------------------

% Header and footer for when a page split occurs within a problem environment
\newcommand{\enterProblemHeader}[1]{
\nobreak\extramarks{#1}{#1 continued on next page\ldots}\nobreak
\nobreak\extramarks{#1 (continued)}{#1 continued on next page\ldots}\nobreak
}

% Header and footer for when a page split occurs between problem environments
\newcommand{\exitProblemHeader}[1]{
\nobreak\extramarks{#1 (continued)}{#1 continued on next page\ldots}\nobreak
\nobreak\extramarks{#1}{}\nobreak
}

\setcounter{secnumdepth}{0} % Removes default section numbers
\newcounter{homeworkProblemCounter} % Creates a counter to keep track of the number of problems

\newcommand{\homeworkProblemName}{}
\newenvironment{homeworkProblem}[1][Problem \arabic{homeworkProblemCounter}]{ % Makes a new environment called homeworkProblem which takes 1 argument (custom name) but the default is "Problem #"
\stepcounter{homeworkProblemCounter} % Increase counter for number of problems
\renewcommand{\homeworkProblemName}{#1} % Assign \homeworkProblemName the name of the problem
\section{\homeworkProblemName} % Make a section in the document with the custom problem count
\enterProblemHeader{\homeworkProblemName} % Header and footer within the environment
}{
\exitProblemHeader{\homeworkProblemName} % Header and footer after the environment
}

\newcommand{\problemAnswer}[1]{ % Defines the problem answer command with the content as the only argument
\noindent\framebox[\columnwidth][c]{\begin{minipage}{0.98\columnwidth}#1\end{minipage}} % Makes the box around the problem answer and puts the content inside
}

\newcommand{\homeworkSectionName}{}
\newenvironment{homeworkSection}[1]{ % New environment for sections within homework problems, takes 1 argument - the name of the section
\renewcommand{\homeworkSectionName}{#1} % Assign \homeworkSectionName to the name of the section from the environment argument
\subsection{\homeworkSectionName} % Make a subsection with the custom name of the subsection
\enterProblemHeader{\homeworkProblemName\ [\homeworkSectionName]} % Header and footer within the environment
}{
\enterProblemHeader{\homeworkProblemName} % Header and footer after the environment
}

%----------------------------------------------------------------------------------------
%	NAME AND CLASS SECTION
%----------------------------------------------------------------------------------------

\newcommand{\hmwkTitle}{Assignment\ \#1} % Assignment title
\newcommand{\hmwkDueDate}{Thursday,\ January\ 26,\ 2017} % Due date
\newcommand{\hmwkClass}{ Introduction to Web Science} % Course/class
\newcommand{\hmwkClassTime}{} % Class/lecture time
\newcommand{\hmwkClassInstructor}{Dr. Nelson} % Teacher/lecturer
\newcommand{\hmwkAuthorName}{Udochukwu Nweke} % Your name

%----------------------------------------------------------------------------------------
%	TITLE PAGE
%----------------------------------------------------------------------------------------

\title{
\vspace{2in}
\textmd{\textbf{\hmwkClass:\ \hmwkTitle}}\\
\normalsize\vspace{0.1in}\small{Due\ on\ \hmwkDueDate}\\
\vspace{0.1in}\large{\textit{\hmwkClassInstructor\ \hmwkClassTime}}
\vspace{3in}
}

\author{\textbf{\hmwkAuthorName}}
\date{} % Insert date here if you want it to appear below your name

%----------------------------------------------------------------------------------------

\begin{document}

\maketitle

%----------------------------------------------------------------------------------------
%	TABLE OF CONTENTS
%----------------------------------------------------------------------------------------

%\setcounter{tocdepth}{1} % Uncomment this line if you don't want subsections listed in the ToC

\newpage
\tableofcontents
\newpage

%----------------------------------------------------------------------------------------
%	PROBLEM 1
%----------------------------------------------------------------------------------------

% To have just one problem per page, simply put a \clearpage after each problem

\begin{homeworkProblem}

Demonstrate that you know how to use ``curl'' well enough to correctly POST data to a form. Show that the HTML response that is returned is ``correct''. That is the server should take the arguments you POSTed and build a response accordingly. Save the HTML response to a file and then view that file in a browser and take a screen shot.



\perlscript{p1}{Shows curl Demonstration Script With Highlighting}

The Listing 1. shows how I used curl command to submit data to the server through a POST method by using --data option. In listing 1, the number 17 submitted to the server at address: ``http://www.numberempire.com
/primenumbers.php''  is printed by the web application.


\begin{figure}
    \centering
    %\textbf{Your title}\par\medskip
    \includegraphics[scale=0.3]{curlDemo}
    \caption{Showing the value 17 correctly received from the server}
\end{figure}


\end{homeworkProblem}

%----------------------------------------------------------------------------------------
%	PROBLEM 2
%----------------------------------------------------------------------------------------

\begin{homeworkProblem}
Write a Python code that:


1. takes as a command line argument a web page


2. extracts all the links from the page


3. lists all the links that result in PDF files, and prints out the bytes for each of the links. ( note: be sure to follow all the redirects until the link terminates with a ``200 ok''.)


4. Show that the program works on 3 different URIs, one of which needs to be:

   http://www.cs.odu.edu/~mln/teaching/cs532-s17/test/pdf/html


\perlscript{p2}{Shows curl Demonstration Script With Highlighting}
I extracted the command line arguments with the python sys.argv array. Next, a the function called \textit{derefURL} extracts the HTML content using curl with the -L option to follow redirects. I used BeautifulSoup to parse the HTML content and extract links with function \textit{getLinks}. I made a HEAD HTML request using curl -I option to process the extracted URLs in other to get the PDF files and bytesize (function \textit{getPDFLinks and getHeadAttr}). The following below is a sample output from ``http://www.cs.odu.edu/~mweigle/Main/PubsByYear''. I have included all the ouput (pdflinks.txt, pdflink1.txt and pdflinks2.txt) in my submission.

\begin{verbatim}
webpage: http://www.cs.odu.edu/~mweigle/Main/PubsByYear
http://www.cs.odu.edu/~mweigle/files/CV.pdf
	7 bytesize: 93364
	type: PDF

http://www.cs.odu.edu/~mweigle/papers/pardue-vis16-2pg-poster.pdf
	17 bytesize: 583877
	type: PDF

http://www.cs.odu.edu/~mweigle/papers/pardue-vis16-poster.pdf
	18 bytesize: 614446
	type: PDF

http://www.cs.odu.edu/~mweigle/papers/aturban-tpdl15.pdf
	33 bytesize: 622981
	type: PDF

http://www.cs.odu.edu/~mln/pubs/tpdl-2015/tpdl-2015-stories.pdf
	35 bytesize: 1274604
	type: PDF

http://www.cs.odu.edu/~mln/pubs/tpdl-2015/tpdl-2015-off-topic.pdf
	37 bytesize: 4308768
	type: PDF
\end{verbatim}

\end{homeworkProblem}



\begin{homeworkProblem}
Consider the "bow-tie" graph in the Broder et al. paper (fig 9):
http://www9.org/w9cdrom/160/160.html

Now consider the following graph:
\begin{verbatim}
	A --> B
	B --> C
    C --> D
    C --> A
    C --> G
    E --> F
    G --> C
    G --> H
    I --> H
    I --> K
    L --> D
    M --> A
    M --> N
    N --> D
    O --> A
    P --> G 
    
\end{verbatim}
For the above graph, give the values for:

IN: {M, P, O}
 
SCC: {A, B, C, G}

OUT: {D, H}

Tendrils: {L, I, K} 

Tubes: {N}

Disconnected: {E, F}

\begin{figure}
    \centering
    %\textbf{Your title}\par\medskip
    \includegraphics[scale=0.3]{graph}
    \caption{bow-tie graph}
\end{figure}

SCC: This is the Strongly Connected Component and it is refers to all the links that are connected to one another along a directed link. Hence, from our graph (http://www9.org/w9cdrom/160/160.html).

 SCC: {A, B, C, D}


IN: These are points that can be reached by SCC but cannot be reached from SCC (http://www9.org/w9cdrom/160/160.html)

IN: {M, P, O}

OUT: These are the pages that can be accessed from the SCC but they do not have any link back to the SCC (http://www9.org/w9cdrom/160/160.html)

OUT: {D,H}

TENDRIL: These are pages that cannot get to the SCC and they cannot be cannot be reached from SCC as well. (http://www9.org/w9cdrom/160/160.html)

TENDRIL: {L, I, K}

TUBES: Pages that have in-links from IN or other pages in Tubes and out-links to pages in Tubes or OUT.
(http://www.harding.edu/fmccown/classes/comp475-s13/web-structure-homework.pdf)

Tube: {N}

DISCONNECTED: Pages that have no in-links from any other components and no out-links to other components.
These pages may be linked to each other. 

(http://www.harding.edu/fmccown/classes/
comp475-s13/web-structure-homework.pdf)

DISCONNECTED {E, F}


\end{homeworkProblem}

%--------------------------------------------------------------------------------
\end{document}

