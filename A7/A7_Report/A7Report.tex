%%%%%%%%%%%%%%%%%%%%%%%%%%%%%%%%%%%%%%%%%
% Programming/Coding Assignment
% LaTeX Template
%
% This template has been downloaded from:
% http://www.latextemplates.com
%
% Original author:
% Ted Pavlic (http://www.tedpavlic.com)
%
% Note:
% The \lipsum[#] commands throughout this template generate dummy text
% to fill the template out. These commands should all be removed when 
% writing assignment content.
%
% This template uses a Perl script as an example snippet of code, most other
% languages are also usable. Configure them in the "CODE INCLUSION 
% CONFIGURATION" section.
%
%%%%%%%%%%%%%%%%%%%%%%%%%%%%%%%%%%%%%%%%%

%----------------------------------------------------------------------------------------
%	PACKAGES AND OTHER DOCUMENT CONFIGURATIONS
%----------------------------------------------------------------------------------------

\documentclass{article}
\usepackage{fancyhdr} % Required for custom headers
\usepackage{lastpage} % Required to determine the last page for the footer
\usepackage{extramarks} % Required for headers and footers
\usepackage[usenames,dvipsnames]{color} % Required for custom colors
\usepackage{graphicx} % Required to insert images
\usepackage{caption}
\usepackage{listings} % Required for insertion of code
\usepackage{courier} % Required for the courier font
\usepackage{lipsum} % Used for inserting dummy 'Lorem ipsum' text into the template
\usepackage[colorlinks=true,linkcolor=black,anchorcolor=black,citecolor=black,menucolor=black,runcolor=black,urlcolor=black,bookmarks=true]{hyperref}
\usepackage[table,svgnames]{xcolor}
\usepackage{tabularx}
\usepackage{booktabs}
\usepackage{natbib}

% Margins
\topmargin=-0.45in
\evensidemargin=0in
\oddsidemargin=0in
\textwidth=6.5in
\textheight=9.0in
\headsep=0.25in

\linespread{1.1} % Line spacing

% Set up the header and footer
\pagestyle{fancy}
\lhead{\hmwkAuthorName} % Top left header
\chead{\hmwkClass\ (\hmwkClassInstructor\ \hmwkClassTime): \hmwkTitle} % Top center head
\rhead{\firstxmark} % Top right header
\lfoot{\lastxmark} % Bottom left footer
\cfoot{} % Bottom center footer
\rfoot{Page\ \thepage\ of\ \protect\pageref{LastPage}} % Bottom right footer
\renewcommand\headrulewidth{0.4pt} % Size of the header rule
\renewcommand\footrulewidth{0.4pt} % Size of the footer rule

\setlength\parindent{0pt} % Removes all indentation from paragraphs

%----------------------------------------------------------------------------------------
%	CODE INCLUSION CONFIGURATION
%----------------------------------------------------------------------------------------

\definecolor{MyDarkGreen}{rgb}{0.0,0.4,0.0} % This is the color used for comments
\lstloadlanguages{Perl} % Load Perl syntax for listings, for a list of other languages supported see: ftp://ftp.tex.ac.uk/tex-archive/macros/latex/contrib/listings/listings.pdf
\lstset{language=Perl, % Use Perl in this example
        frame=single, % Single frame around code
        basicstyle=\small\ttfamily, % Use small true type font
        keywordstyle=[1]\color{Blue}\bf, % Perl functions bold and blue
        keywordstyle=[2]\color{Purple}, % Perl function arguments purple
        keywordstyle=[3]\color{Blue}\underbar, % Custom functions underlined and blue
        identifierstyle=, % Nothing special about identifiers                                         
        commentstyle=\usefont{T1}{pcr}{m}{sl}\color{MyDarkGreen}\small, % Comments small dark green courier font
        stringstyle=\color{Purple}, % Strings are purple
        showstringspaces=false, % Don't put marks in string spaces
        tabsize=5, % 5 spaces per tab
        %
        % Put standard Perl functions not included in the default language here
        morekeywords={rand},
        %
        % Put Perl function parameters here
        morekeywords=[2]{on, off, interp},
        %
        % Put user defined functions here
        morekeywords=[3]{test},
       	%
        morecomment=[l][\color{Blue}]{...}, % Line continuation (...) like blue comment
        numbers=left, % Line numbers on left
        firstnumber=1, % Line numbers start with line 1
        numberstyle=\tiny\color{Blue}, % Line numbers are blue and small
        stepnumber=5 % Line numbers go in steps of 5
}

% Creates a new command to include a perl script, the first parameter is the filename of the script (without .pl), the second parameter is the caption




%----------------------------------------------------------------------------------------
%	DOCUMENT STRUCTURE COMMANDS
%	Skip this unless you know what you're doing
%----------------------------------------------------------------------------------------

% Header and footer for when a page split occurs within a problem environment
\newcommand{\enterProblemHeader}[1]{
\nobreak\extramarks{#1}{#1 continued on next page\ldots}\nobreak
\nobreak\extramarks{#1 (continued)}{#1 continued on next page\ldots}\nobreak
}

% Header and footer for when a page split occurs between problem environments
\newcommand{\exitProblemHeader}[1]{
\nobreak\extramarks{#1 (continued)}{#1 continued on next page\ldots}\nobreak
\nobreak\extramarks{#1}{}\nobreak
}

\setcounter{secnumdepth}{0} % Removes default section numbers
\newcounter{homeworkProblemCounter} % Creates a counter to keep track of the number of problems

\newcommand{\homeworkProblemName}{}
\newenvironment{homeworkProblem}[1][Problem \arabic{homeworkProblemCounter}]{ % Makes a new environment called homeworkProblem which takes 1 argument (custom name) but the default is "Problem #"
\stepcounter{homeworkProblemCounter} % Increase counter for number of problems
\renewcommand{\homeworkProblemName}{#1} % Assign \homeworkProblemName the name of the problem
\section{\homeworkProblemName} % Make a section in the document with the custom problem count
\enterProblemHeader{\homeworkProblemName} % Header and footer within the environment
}{
\exitProblemHeader{\homeworkProblemName} % Header and footer after the environment
}

\newcommand{\problemAnswer}[1]{ % Defines the problem answer command with the content as the only argument
\noindent\framebox[\columnwidth][c]{\begin{minipage}{0.98\columnwidth}#1\end{minipage}} % Makes the box around the problem answer and puts the content inside
}

\newcommand{\homeworkSectionName}{}
\newenvironment{homeworkSection}[1]{ % New environment for sections within homework problems, takes 1 argument - the name of the section
\renewcommand{\homeworkSectionName}{#1} % Assign \homeworkSectionName to the name of the section from the environment argument
\subsection{\homeworkSectionName} % Make a subsection with the custom name of the subsection
\enterProblemHeader{\homeworkProblemName\ [\homeworkSectionName]} % Header and footer within the environment
}{
\enterProblemHeader{\homeworkProblemName} % Header and footer after the environment
}

%----------------------------------------------------------------------------------------
%	NAME AND CLASS SECTION
%----------------------------------------------------------------------------------------

\newcommand{\hmwkTitle}{A7} % Assignment title
\newcommand{\hmwkDueDate}{Thursday,\ April\ 6,\ 2017} % Due date
\newcommand{\hmwkClass}{\ INTRO. TO WEB SCIENCE:\ CS 532} % Course/class
\newcommand{\hmwkClassTime}{} % Class/lecture time
\newcommand{\hmwkClassInstructor}{Dr. Nelson} % Teacher/lecturer
\newcommand{\hmwkAuthorName}{Udochukwu Nweke} % Your name

%----------------------------------------------------------------------------------------
%	TITLE PAGE
%----------------------------------------------------------------------------------------

\title{
\vspace{2in}
\textmd{\textbf{\hmwkClass:\ \hmwkTitle}}\\
\normalsize\vspace{0.1in}\small{Due\ on\ \hmwkDueDate}\\
\vspace{0.1in}\large{\textit{\hmwkClassInstructor\ \hmwkClassTime}}
\vspace{3in}
}

\author{\textbf{\hmwkAuthorName}}
\date{} % Insert date here if you want it to appear below your name

%----------------------------------------------------------------------------------------

\begin{document}

\maketitle

%----------------------------------------------------------------------------------------
%	TABLE OF CONTENTS
%----------------------------------------------------------------------------------------

%\setcounter{tocdepth}{1} % Uncomment this line if you don't want subsections listed in the ToC

\newpage
\tableofcontents
\newpage

%----------------------------------------------------------------------------------------
%	PROBLEM 1
%----------------------------------------------------------------------------------------

% To have just one problem per page, simply put a \clearpage after each problem

\begin{homeworkProblem}

\lstinputlisting[caption=MovieLens Solutions Code, language=python]{P1-4.py}

\lstinputlisting[caption=MovieLens Code, language=python]{recommendations.py}

The goal of this project is to use the basic recommendation principles
we have learned for user-collected data. You will modify the code
given to you which performs movie recommendations from the MovieLense
data sets.\\

The MovieLense data sets were collected by the GroupLens Research
Project at the University of Minnesota during the seven-month period
from September 19th, 1997 through April 22nd, 1998.  We are using the 
``100k dataset''; available for download from:
\url{http://grouplens.org/datasets/movielens/100k/}

There are three files which we will use:

1.  u.data: 100,000 ratings by 943 users on 1,682 movies. Each
user has rated at least 20 movies. Users and items are numbered
consecutively from 1. The data is randomly ordered. This is a tab
separated list of \\

\begin{verbatim}
user id | item id | rating | timestamp
\end{verbatim}
The time stamps are unix seconds since 1/1/1970 UTC.

1.  Find 3 users who are closest to you in terms of age, 
gender, and occupation.  For each of those 3 users:

- what are their top 3 favorite films?
- bottom 3 least favorite films?

Based on the movie values in those 6 tables (3 users X (favorite +
least)), choose a user that you feel is most like you.  Feel 
free to note any outliers (e.g., ``I mostly identify with user 123,
except I did not like ``Ghost'' at all'').  

This user is the ``substitute you''. \\



\textbf{Solution 1:}\\

In order to find 3 users who are closest to me in terms of age, gender, and occupation: I read the user data into a dictionary, with the key as userid and values represented by a vector containing age, gender, and occupation.\\

I created a mapping of Gender: Male represented by 1,  Female represented by 0 and occupation: the first occupation (Technician) represented by  0, second occupation (Other) represented by  1, third occupation (Writer)  -2,  student  -5,  etc. \\

I used \textit{euclideanDistance()} in listing 1 to compute euclidean distance between my vector representing my age, gender and occupation [32, 0, 5] and each user, and \textit{getClosestTriple()}(Listing 1.) to retrieve the users closest to me. The result is seen in Table \ref{tab:3 Users closest to me}\\ 

\begin{table}[h!]
 \centering
 \caption{My Closest Triple}
  \label{tab:3 Users closest to me}
    \begin{tabular}{ |l| l | l | l| l|}
     \hline
 User id & Age& Gender& Occupation& Euclidean Distance\\
 \hline
 560&32& Male& Student & 1.0\\
 \hline
 350&32&  Male& Student& 1.0\\
 \hline
 890&32& Male& Student& 1.0\\
 \hline
 

   \end{tabular}
 \end{table}


I sorted based on their movie ratings and picked their top 3 ratings and bottom 3 ratings.  The result is seen in Table  \ref{tab:`350'  3 Top Favorite} through Table \ref{tab:`890' 3 Least Favorite}.\\



\begin{table}[h!]
 \centering
 \caption{User `350' Top 3 Favorite Films}
  \label{tab:`350'  3 Top Favorite}
    \begin{tabular}{ | l | l | l|}
     \hline
 Item& Movie-Title& Rating\\
 \hline
 1 &Gone with the Wind (1939)& 5.0\\
 \hline
 2 &Casablanca (1942) &  5.0\\
 \hline
 3 &Empire Strikes Back, The (1980) & 5.0\\
 \hline
 
 \end{tabular}
\end{table}

 

\begin{table}[h!]
 \centering
 \caption{User `350' 3 Least Favorite Films}
  \label{tab:`350'  3 Least Favorite}
    \begin{tabular}{ | l | l | l|}
     \hline
 Item& Movie-Title& Rating\\
 \hline
 1 &Hunt for Red October, The (1990) & 2.0\\
 \hline
 2 &M*A*S*H (1970) &  2.0\\
 \hline
 3 &Contact (1997)& 3.0\\
 \hline
 \end{tabular}
\end{table}



\begin{table}[h!]
 \centering
 \caption{User `560' Top 3 Favorite Films}
  \label{tab:`560'  3 Top Favorite}
    \begin{tabular}{ | l | l | l|}
     \hline
 Item& Movie& Rating\\
 \hline
 1 &Star Wars (1977) & 5.0\\
 \hline
 2 &Chinatown (1974) &  5.0\\
 \hline
 3 &: Alien (1979) & 5.0\\
 \hline
 
 \end{tabular}
\end{table}



\begin{table}[h!]
 \centering
 \caption{User `560'  3 Least Favorite Films}
  \label{tab:`560'  3 Least Favorite}
    \begin{tabular}{ | l | l | l|}
     \hline
 Item& Movie-Title& Rating\\
 \hline
 1 &Event Horizon (1997)& 1.0\\
 \hline
 2 &Kids in the Hall: Brain Candy (1996) &  1.0\\
 \hline
 3 &Bed of Roses (1996)& 1.0\\
 \hline
 
 \end{tabular}
\end{table}



\begin{table}[h!]
 \centering
 \caption{User `890' Top 3 Favorite Films}
  \label{tab:`890' Top 3 Favorite}
    \begin{tabular}{ | l | l | l|}
     \hline
 Item& Movie-Title& Rating\\
 \hline
 1 &To Kill a Mockingbird (1962) & 5.0\\
 \hline
 2 &One Flew Over the Cuckoo's Nest (1975) &  5.0\\
 \hline
 3 &Empire Strikes Back, The (1980) & 5.0\\
 \hline
 
 \end{tabular}
\end{table}



\begin{table}[h!]
 \centering
 \caption{User `890' 3 Least Favorite Films}
  \label{tab:`890' 3 Least Favorite}
    \begin{tabular}{ | l | l | l|}
     \hline
 Item& Movie-Title& Rating\\
 \hline
 1 &Star Trek: The Motion Picture (1979) & 1.0\\
 \hline
 2 &Ref, The (1994) &  1.0\\
 \hline
 3 &Batman (1989)& 1.0\\
 \hline
 
 \end{tabular}
\end{table}


The substitute me will be user `350'.The only difference is that I didn't like M*A*S*H (1970) at all. I will probably rate it 1.0 if I had to.




\end{homeworkProblem}
%----------------------------------------------------------------------------------------
% PROBLEM 2
%----------------------------------------------------------------------------------------

\begin{homeworkProblem}

Which 5 users are most correlated to the substitute you? Which
5 users are least correlated (i.e., negative correlation)?\\

\textbf{Solution 2:}\\
 
In order to compute the users that are least correlated to the substitute me, which is user `350', I modified \textit{topMatches()} in listing 2 by adding option to reverse to get most correlated users. The \textit{topMatches()} gives the least correlated users by default. The result is seen in Table \ref{tab:5 Most Correalated Users} and Table \ref{tab:5 Least Correalated Users}\\


 \begin{table}[h!]
  \centering
  \caption{5 Users Most Correlated to User `350'}
   \label{tab:5 Most Correalated Users}
     \begin{tabular}{ | l | l | l| }
      \hline
 Item& User& Correlation Coefficient\\
 \hline
 1 & 544 & 1.0\\
 \hline
 2 & 939 &  1.0\\
 \hline
 3 &915& 1.0\\
 \hline
 4 &904& 1.0\\
 \hline
 5 &888& 1.0\\
 \hline
 
     \end{tabular}
  \end{table}

 \begin{table}[h!]
  \centering
  \caption{5 Users Least Correlated to User `350'}
   \label{tab:5 Least Correalated Users}
     \begin{tabular}{ | l | l | l| }
      \hline
 Item& User& Correlation Coefficient\\
 \hline
 1 &133  & -1.0\\
 \hline
 2 &166 &  -1.0\\
 \hline
 3 &17& -1.0\\
 \hline
 4 &172& -1.0\\
 \hline
 5 &190& -1.0\\
 \hline
 
    \end{tabular}
  \end{table}
\end{homeworkProblem}

%----------------------------------------------------------------------------------------
% PROBLEM 3
%----------------------------------------------------------------------------------------
\begin{homeworkProblem}

Compute ratings for all the films that the substitute you
have not seen.  Provide a list of the top 5 recommendations for films
that the substitute you should see.  Provide a list of the bottom
5 recommendations (i.e., films the substitute you is almost certain
to hate).\\

\textbf{Solution 3:}\\

In order to recommend movies to  user 350, I used \textit{getRecommendations()} in listing 2 to compute the most recommended movies and least recommended movies by taking into account the movies  user 350 have not seen. Table \ref{tab:5 Most Recommended Movies} and Table \ref{tab:5 Least Recommended Movies} shows the top and least recommended movies for user 350 respectively. 


 \begin{table}[h!]
  \centering
  \caption{5 Most Recommended Movies for User `350'}
   \label{tab:5 Most Recommended Movies}
     \begin{tabular}{ | l | l | l| }
      \hline
 Item& Movie-Title& Rating\\
 \hline
 1 &They Made Me a Criminal (1939) & 5.0\\
 \hline
 2 &The Deadly Cure (1996) &  5.0\\
 \hline
 3 &Someone Else's America (1995)& 5.0\\
 \hline
 4 &Santa with Muscles (1996)& 5.0\\
 \hline
 5 &Prefontaine (1997)& 5.0\\
 \hline
 
     \end{tabular}
  \end{table}

 \begin{table}[h!]
  \centering
  \caption{5 Least Recommendation Movies for User `350'}
   \label{tab:5 Least Recommended Movies}
     \begin{tabular}{ | l | l | l| }
      \hline
 Item& Movie-Title& Rating\\
 \hline
 1 &3 Ninjas: High Noon At Mega Mountain (1998) & 1.0\\
 \hline
 2 &Amityville 1992: It's About Time (1992) &  1.0\\
 \hline
 3 &Amityville: A New Generation (1993)& 1.0\\
 \hline
 4 &Amityville: Dollhouse (1996)& 1.0\\
 \hline
 5 &B*A*P*S (1997)& 1.0\\
 \hline
 
 \end{tabular}
\end{table}

\end{homeworkProblem}


%----------------------------------------------------------------------------------------
% PROBLEM 4
%----------------------------------------------------------------------------------------
\begin{homeworkProblem}

Choose your (the real you, not the substitute you) favorite and
least favorite film from the data.  For each film, generate a list
of the top 5 most correlated and bottom 5 least correlated films.
Based on your knowledge of the resulting films, do you agree with
the results?  In other words, do you personally like / dislike
the resulting films?\\

\textbf{Solution 4:}\\

I chose the  Beautician and the Beast, The (1997) as my favorite film and  Steal Big, Steal Little (1995) as my least favorite.  This because I have seen both movies.

In order to generate a list of the top most correlated and bottom least correlated movies for my favorite and least favorite movies, I used \textit{calculateSimilarItems()} in listing 2. The result is seen from Table \ref{tab:5 Most Correlated Movies to Beautician and the Beast} through Table \ref{tab:5 Least Correlated Movies to Steal Big, Steal Little}\\ 

I agree with the resulting films in Table \ref{tab:5 Most Correlated Movies to Beautician and the Beast} and Table \ref{tab:5 Least Correlated Movies Beautician and the Beast}. Except for Twin Town in Table \ref{tab:5 Most Correlated Movies to Beautician and the Beast}. I would have included it in least correlated movies in Table \ref{tab:5 Least Correlated Movies Beautician and the Beast}.

For my least favorite film, I totally agree with the resulting films in Table \ref{tab:5 Most Correlated Movies to Steal Big, Steal Little} and Table \ref{tab:5 Least Correlated Movies to Steal Big, Steal Little} .  

\begin{table}[h!]
  \centering
  \caption{5  Most Correlated Movies to `Beautician and the Beast'}
   \label{tab:5 Most Correlated Movies to Beautician and the Beast}
     \begin{tabular}{ | l | l | l| }
      \hline
 Item& Movie-Title& Similarity Rate\\
 \hline
 1 &Wild America (1997) & 1.0\\
 \hline
 2 &Welcome To Sarajevo (1997) &  1.0\\
 \hline
 3 &Warriors of Virtue (1997)'& 1.0\\
 \hline
 4 &Twisted (1996)& 1.0\\
 \hline
 5 &Twin Town (1997)& 1.0\\
 \hline
 
     \end{tabular}
  \end{table}

\begin{table}[h!]
  \centering
  \caption{5 Least Correlated Movies To Beautician and the Beast}
   \label{tab:5 Least Correlated Movies Beautician and the Beast}
     \begin{tabular}{ | l | l | l| }
      \hline
 Item& Movie-Title& Similarity Rate\\
 \hline
 1 &Til There Was You (1997)) & 0\\
 \hline
 2 &8 Seconds (1994) &  0\\
 \hline
 3 &A Chef in Love (1996)& 0\\
 \hline
 4 &Above the Rim (1994)& 0\\
 \hline
 5 &Across the Sea of Time (1995)& 0\\
 \hline
 
     \end{tabular}
  \end{table}

  \begin{table}[h!]
  \centering
  \caption{5 Most Correlated Movies To Steal Big, Steal Little (1995)}
   \label{tab:5 Most Correlated Movies to Steal Big, Steal Little}
     \begin{tabular}{ | l | l | l| }
      \hline
 Item& Movie-Title& Similarity Rate\\
 \hline
 1 &Yankee Zulu (1994) & 1.0\\
 \hline
 2 &Wyatt Earp (1994) &  1.0\\
 \hline
 3 &Woman in Question, The (1950)& 1.0\\
 \hline
 4 &Withnail and I (1987)& 1.0\\
 \hline
 5 &Wishmaster (1997)& 1.0\\
 \hline
 
     \end{tabular}
  \end{table}

 \begin{table}[h!]
  \centering
  \caption{5 Least Correlated Movies To Steal Big, Steal Little (1995)}
   \label{tab:5 Least Correlated Movies to Steal Big, Steal Little}
     \begin{tabular}{ | l | l | l| }
      \hline
 Item& Movie-Title& Similarity Rate\\
 \hline
 1 &Til There Was You (1997) & 0\\
 \hline
 2 &87 (1997) &  0\\
 \hline
 3 &3 Ninjas: High Noon At Mega Mountain (1998)& 0\\
 \hline
 4 &39 Steps, The (1935)& 0\\
 \hline
 5 &8 Heads in a Duffel Bag (1997)& 0\\
 \hline
 
     \end{tabular}
  \end{table}



















\end{homeworkProblem}
\nocite{*}
\bibliographystyle{plain}
\bibliography{A7Ref}

\end{document}