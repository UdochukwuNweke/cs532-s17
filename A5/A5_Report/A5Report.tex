%%%%%%%%%%%%%%%%%%%%%%%%%%%%%%%%%%%%%%%%%
% Programming/Coding Assignment
% LaTeX Template
%
% This template has been downloaded from:
% http://www.latextemplates.com
%
% Original author:
% Ted Pavlic (http://www.tedpavlic.com)
%
% Note:
% The \lipsum[#] commands throughout this template generate dummy text
% to fill the template out. These commands should all be removed when 
% writing assignment content.
%
% This template uses a Perl script as an example snippet of code, most other
% languages are also usable. Configure them in the "CODE INCLUSION 
% CONFIGURATION" section.
%
%%%%%%%%%%%%%%%%%%%%%%%%%%%%%%%%%%%%%%%%%

%----------------------------------------------------------------------------------------
%	PACKAGES AND OTHER DOCUMENT CONFIGURATIONS
%----------------------------------------------------------------------------------------

\documentclass{article}
\usepackage{fancyhdr} % Required for custom headers
\usepackage{lastpage} % Required to determine the last page for the footer
\usepackage{extramarks} % Required for headers and footers
\usepackage[usenames,dvipsnames]{color} % Required for custom colors
\usepackage{graphicx} % Required to insert images
\usepackage{afterpage}
\usepackage{caption}
\usepackage{subcaption}
\usepackage{listings} % Required for insertion of code
\usepackage{courier} % Required for the courier font
\usepackage{lipsum} % Used for inserting dummy 'Lorem ipsum' text into the template
\usepackage[colorlinks=true,linkcolor=black,anchorcolor=black,citecolor=black,menucolor=black,runcolor=black,urlcolor=black,bookmarks=true]{hyperref}
\usepackage[table,svgnames]{xcolor}
\usepackage{tabularx}
\usepackage{booktabs}
\usepackage{natbib}

% Margins
\topmargin=-0.45in
\evensidemargin=0in
\oddsidemargin=0in
\textwidth=6.5in
\textheight=9.0in
\headsep=0.25in

\linespread{1.1} % Line spacing

% Set up the header and footer
\pagestyle{fancy}
\lhead{\hmwkAuthorName} % Top left header
\chead{\hmwkClass\ (\hmwkClassInstructor\ \hmwkClassTime): \hmwkTitle} % Top center head
\rhead{\firstxmark} % Top right header
\lfoot{\lastxmark} % Bottom left footer
\cfoot{} % Bottom center footer
\rfoot{Page\ \thepage\ of\ \protect\pageref{LastPage}} % Bottom right footer
\renewcommand\headrulewidth{0.4pt} % Size of the header rule
\renewcommand\footrulewidth{0.4pt} % Size of the footer rule

\setlength\parindent{0pt} % Removes all indentation from paragraphs

%----------------------------------------------------------------------------------------
%	CODE INCLUSION CONFIGURATION
%----------------------------------------------------------------------------------------

\definecolor{MyDarkGreen}{rgb}{0.0,0.4,0.0} % This is the color used for comments
\lstloadlanguages{Perl} % Load Perl syntax for listings, for a list of other languages supported see: ftp://ftp.tex.ac.uk/tex-archive/macros/latex/contrib/listings/listings.pdf
\lstset{language=Perl, % Use Perl in this example
        frame=single, % Single frame around code
        basicstyle=\small\ttfamily, % Use small true type font
        keywordstyle=[1]\color{Blue}\bf, % Perl functions bold and blue
        keywordstyle=[2]\color{Purple}, % Perl function arguments purple
        keywordstyle=[3]\color{Blue}\underbar, % Custom functions underlined and blue
        identifierstyle=, % Nothing special about identifiers                                         
        commentstyle=\usefont{T1}{pcr}{m}{sl}\color{MyDarkGreen}\small, % Comments small dark green courier font
        stringstyle=\color{Purple}, % Strings are purple
        showstringspaces=false, % Don't put marks in string spaces
        tabsize=5, % 5 spaces per tab
        %
        % Put standard Perl functions not included in the default language here
        morekeywords={rand},
        %
        % Put Perl function parameters here
        morekeywords=[2]{on, off, interp},
        %
        % Put user defined functions here
        morekeywords=[3]{test},
       	%
        morecomment=[l][\color{Blue}]{...}, % Line continuation (...) like blue comment
        numbers=left, % Line numbers on left
        firstnumber=1, % Line numbers start with line 1
        numberstyle=\tiny\color{Blue}, % Line numbers are blue and small
        stepnumber=5 % Line numbers go in steps of 5
}

% Creates a new command to include a perl script, the first parameter is the filename of the script (without .pl), the second parameter is the caption




%----------------------------------------------------------------------------------------
%	DOCUMENT STRUCTURE COMMANDS
%	Skip this unless you know what you're doing
%----------------------------------------------------------------------------------------

% Header and footer for when a page split occurs within a problem environment
\newcommand{\enterProblemHeader}[1]{
\nobreak\extramarks{#1}{#1 continued on next page\ldots}\nobreak
\nobreak\extramarks{#1 (continued)}{#1 continued on next page\ldots}\nobreak
}

% Header and footer for when a page split occurs between problem environments
\newcommand{\exitProblemHeader}[1]{
\nobreak\extramarks{#1 (continued)}{#1 continued on next page\ldots}\nobreak
\nobreak\extramarks{#1}{}\nobreak
}

\setcounter{secnumdepth}{0} % Removes default section numbers
\newcounter{homeworkProblemCounter} % Creates a counter to keep track of the number of problems

\newcommand{\homeworkProblemName}{}
\newenvironment{homeworkProblem}[1][Problem \arabic{homeworkProblemCounter}]{ % Makes a new environment called homeworkProblem which takes 1 argument (custom name) but the default is "Problem #"
\stepcounter{homeworkProblemCounter} % Increase counter for number of problems
\renewcommand{\homeworkProblemName}{#1} % Assign \homeworkProblemName the name of the problem
\section{\homeworkProblemName} % Make a section in the document with the custom problem count
\enterProblemHeader{\homeworkProblemName} % Header and footer within the environment
}{
\exitProblemHeader{\homeworkProblemName} % Header and footer after the environment
}

\newcommand{\problemAnswer}[1]{ % Defines the problem answer command with the content as the only argument
\noindent\framebox[\columnwidth][c]{\begin{minipage}{0.98\columnwidth}#1\end{minipage}} % Makes the box around the problem answer and puts the content inside
}

\newcommand{\homeworkSectionName}{}
\newenvironment{homeworkSection}[1]{ % New environment for sections within homework problems, takes 1 argument - the name of the section
\renewcommand{\homeworkSectionName}{#1} % Assign \homeworkSectionName to the name of the section from the environment argument
\subsection{\homeworkSectionName} % Make a subsection with the custom name of the subsection
\enterProblemHeader{\homeworkProblemName\ [\homeworkSectionName]} % Header and footer within the environment
}{
\enterProblemHeader{\homeworkProblemName} % Header and footer after the environment
}

%----------------------------------------------------------------------------------------
%	NAME AND CLASS SECTION
%----------------------------------------------------------------------------------------

\newcommand{\hmwkTitle}{A5} % Assignment title
\newcommand{\hmwkDueDate}{Thursday,\ March\ 16,\ 2017} % Due date
\newcommand{\hmwkClass}{\ INTRO. TO WEB SCIENCE:\ CS 532} % Course/class
\newcommand{\hmwkClassTime}{} % Class/lecture time
\newcommand{\hmwkClassInstructor}{Dr. Nelson} % Teacher/lecturer
\newcommand{\hmwkAuthorName}{Udochukwu Nweke} % Your name

%----------------------------------------------------------------------------------------
%	TITLE PAGE
%----------------------------------------------------------------------------------------

\title{
\vspace{2in}
\textmd{\textbf{\hmwkClass:\ \hmwkTitle}}\\
\normalsize\vspace{0.1in}\small{Due\ on\ \hmwkDueDate}\\
\vspace{0.1in}\large{\textit{\hmwkClassInstructor\ \hmwkClassTime}}
\vspace{3in}
}

\author{\textbf{\hmwkAuthorName}}
\date{} % Insert date here if you want it to appear below your name

%----------------------------------------------------------------------------------------

\begin{document}

\maketitle

%----------------------------------------------------------------------------------------
%	TABLE OF CONTENTS
%----------------------------------------------------------------------------------------

%\setcounter{tocdepth}{1} % Uncomment this line if you don't want subsections listed in the ToC

\newpage
\tableofcontents
\newpage

%----------------------------------------------------------------------------------------
%	PROBLEM 1
%----------------------------------------------------------------------------------------

% To have just one problem per page, simply put a \clearpage after each problem

\begin{homeworkProblem}

\lstinputlisting[caption=Girvan and Newman Algorithm Implementation Code, language=python]{P1.py}

\begin{figure}
    \caption{Girvan and Newman Karate club split graph}
    \centering
    \label{Karate club graph before split}
    \begin{subfigure}[b]{0.3\textwidth}
        \includegraphics[width=\textwidth]{karateClub-beforesplit}
        \caption{Karate club graph before split}
    \end{subfigure}
    ~ %add desired spacing between images, e. g. ~, \quad, \qquad, \hfill etc. 
      %(or a blank line to force the subfigure onto a new line)
    \begin{subfigure}[b]{0.3\textwidth}
        \includegraphics[width=\textwidth]{karateClub-postsplit}
        \caption{Karate club graph after split}
    \end{subfigure}
\end{figure}
    ~ %add desired spacing between images, e. g. ~, \quad, \qquad, \hfill etc. 
    %(or a blank line to force the subfigure onto a new line)
\begin{figure}
    \caption{Iterative application of Girvin and Newman Algorithm}
    \centering
    \begin{subfigure}[b]{0.3\textwidth}
        \includegraphics[width=\textwidth]{1-1}
        \caption{G-N-Iteration1-subgraph1}
    \end{subfigure}
    ~ %add desired spacing between images, e. g. ~, \quad, \qquad, \hfill etc. 
    %(or a blank line to force the subfigure onto a new line)
    \begin{subfigure}[b]{0.3\textwidth}
        \includegraphics[width=\textwidth]{2-1}
        \caption{G-N-Iteration2-subgraph1}
    \end{subfigure}
    ~ %add desired spacing between images, e. g. ~, \quad, \qquad, \hfill etc. 
    %(or a blank line to force the subfigure onto a new line)
    \begin{subfigure}[b]{0.3\textwidth}
        \includegraphics[width=\textwidth]{3-1}
        \caption{Girvan-Newman Iteration3-subgraph1}
    \end{subfigure}
    ~ %add desired spacing between images, e. g. ~, \quad, \qquad, \hfill etc. 
    %(or a blank line to force the subfigure onto a new line)
    \begin{subfigure}[b]{0.3\textwidth}
        \includegraphics[width=\textwidth]{4-1}
        \caption{G-N-Iteration4-subgraph1}
    \end{subfigure}

    ~ %add desired spacing between images, e. g. ~, \quad, \qquad, \hfill etc. 
    %(or a blank line to force the subfigure onto a new line)
    \begin{subfigure}[b]{0.3\textwidth}
        \includegraphics[width=\textwidth]{5-1}
        \caption{G-N-Iteration5-subgraph1}
    \end{subfigure}
    ~ %add desired spacing between images, e. g. ~, \quad, \qquad, \hfill etc. 
    %(or a blank line to force the subfigure onto a new line)
    \begin{subfigure}[b]{0.3\textwidth}
        \includegraphics[width=\textwidth]{6-1}
        \caption{G-N- Iteration6-subgraph1}
    \end{subfigure}
    ~ %add desired spacing between images, e. g. ~, \quad, \qquad, \hfill etc. 
    %(or a blank line to force the subfigure onto a new line)
    \begin{subfigure}[b]{0.3\textwidth}
        \includegraphics[width=\textwidth]{7-1}
        \caption{G-N-Iteration7-subgraph1}
    \end{subfigure}

    ~ %add desired spacing between images, e. g. ~, \quad, \qquad, \hfill etc. 
    %(or a blank line to force the subfigure onto a new line)
    \begin{subfigure}[b]{0.3\textwidth}
        \includegraphics[width=\textwidth]{8-1}
        \caption{G-N-Iteration8-subgraph1}
    \end{subfigure}
    ~ %add desired spacing between images, e. g. ~, \quad, \qquad, \hfill etc. 
    %(or a blank line to force the subfigure onto a new line)
    \begin{subfigure}[b]{0.3\textwidth}
        \includegraphics[width=\textwidth]{9-1}
        \caption{G-N-Iteration9-subgraph1}
    \end{subfigure}
    ~ %add desired spacing between images, e. g. ~, \quad, \qquad, \hfill etc. 
    %(or a blank line to force the subfigure onto a new line)
    \begin{subfigure}[b]{0.3\textwidth}
        \includegraphics[width=\textwidth]{10-1}
        \caption{G-N-Iteration10-subgraph1}
    \end{subfigure} 
\end{figure}
    ~ %add desired spacing between images, e. g. ~, \quad, \qquad, \hfill etc. 
    %(or a blank line to force the subfigure onto a new line)
\begin{figure}
    \caption{Iterative application of Girvin and Newman Algorithm Graph Split into Two Sub graphs}
    \centering
    \begin{subfigure}[b]{0.3\textwidth}
        \includegraphics[width=\textwidth]{11-1}
        \caption{G-N-Iteration11-subgraph1}
    \end{subfigure}
    ~ %add desired spacing between images, e. g. ~, \quad, \qquad, \hfill etc. 
    %(or a blank line to force the subfigure onto a new line)
    \begin{subfigure}[b]{0.3\textwidth}
        \includegraphics[width=\textwidth]{12-2}
        \caption{G-N-Iteration12-subgraph2}
    \end{subfigure}
\end{figure}
    ~ %add desired spacing between images, e. g. ~, \quad, \qquad, \hfill etc. 
    %(or a blank line to force the subfigure onto a new line)
\begin{figure}
    \caption{Iterative application of Girvin and Newman Algorithm}
    \centering
    \begin{subfigure}[b]{0.3\textwidth}
        \includegraphics[width=\textwidth]{13-2}
        \caption{G-N-Iteration13-subgraph2}
    \end{subfigure}
    ~ %add desired spacing between images, e. g. ~, \quad, \qquad, \hfill etc. 
    %(or a blank line to force the subfigure onto a new line)
    \begin{subfigure}[b]{0.3\textwidth}
        \includegraphics[width=\textwidth]{14-2}
        \caption{G-N-Iteration14-subgraph2}
    \end{subfigure}
    ~ %add desired spacing between images, e. g. ~, \quad, \qquad, \hfill etc. 
    %(or a blank line to force the subfigure onto a new line)
    \begin{subfigure}[b]{0.3\textwidth}
        \includegraphics[width=\textwidth]{15-3}
        \caption{G-N-Iteration15-subgraph3}
    \end{subfigure}
    ~ %add desired spacing between images, e. g. ~, \quad, \qquad, \hfill etc. 
    %(or a blank line to force the subfigure onto a new line)
    \begin{subfigure}[b]{0.3\textwidth}
        \includegraphics[width=\textwidth]{16-3}
        \caption{G-N-Iteration16-subgraph3}
        \label{fig: Iteraion 16-Subgraph3}
    \end{subfigure}

    ~ %add desired spacing between images, e. g. ~, \quad, \qquad, \hfill etc. 
    %(or a blank line to force the subfigure onto a new line)
    \begin{subfigure}[b]{0.3\textwidth}
        \includegraphics[width=\textwidth]{17-3}
        \caption{G-N-Iteration17-subgraph3}
    \end{subfigure}
    ~ %add desired spacing between images, e. g. ~, \quad, \qquad, \hfill etc. 
    %(or a blank line to force the subfigure onto a new line)
    \begin{subfigure}[b]{0.3\textwidth}
        \includegraphics[width=\textwidth]{18-3}
        \caption{G-N-Iteration18-subgraph3}
    \end{subfigure}
    ~ %add desired spacing between images, e. g. ~, \quad, \qquad, \hfill etc. 
    %(or a blank line to force the subfigure onto a new line)
    \begin{subfigure}[b]{0.3\textwidth}
        \includegraphics[width=\textwidth]{19-4}
        \caption{G-N-Iteration19-subgraph4}
        \label{G-N-Iteration19-subgraph4}
    \end{subfigure}
\end{figure} 
    ~ %add desired spacing between images, e. g. ~, \quad, \qquad, \hfill etc. 
    %(or a blank line to force the subfigure onto a new line)


\begin{figure}
    \begin{subfigure}[b]{0.3\textwidth}
        \includegraphics[width=\textwidth]{20-4}
        \caption{G-N-Iteration20-subgraph4}
    \end{subfigure}
    ~ %add desired spacing between images, e. g. ~, \quad, \qquad, \hfill etc. 
    %(or a blank line to force the subfigure onto a new line)
    \begin{subfigure}[b]{0.3\textwidth}
        \includegraphics[width=\textwidth]{21-4}
        \caption{G-N-Iteration21-subgraph4}
    \end{subfigure}
    ~ %add desired spacing between images, e. g. ~, \quad, \qquad, \hfill etc. 
    %(or a blank line to force the subfigure onto a new line)
    \begin{subfigure}[b]{0.3\textwidth}
        \includegraphics[width=\textwidth]{22-4}
        \caption{G-N-Iteration22-subgraph4}
    \end{subfigure}

    ~ %add desired spacing between images, e. g. ~, \quad, \qquad, \hfill etc. 
    %(or a blank line to force the subfigure onto a new line)
    \begin{subfigure}[b]{0.3\textwidth}
        \includegraphics[width=\textwidth]{23-4}
        \caption{G-N-Iteration23-subgraph4}
    \end{subfigure}
    ~ %add desired spacing between images, e. g. ~, \quad, \qquad, \hfill etc. 
    %(or a blank line to force the subfigure onto a new line)
    \begin{subfigure}[b]{0.3\textwidth}
        \includegraphics[width=\textwidth]{24-4}
        \caption{G-N-Iteration24-subgraph4}
    \end{subfigure}
~ %add desired spacing between images, e. g. ~, \quad, \qquad, \hfill etc. 
    %(or a blank line to force the subfigure onto a new line)
    \begin{subfigure}[b]{0.3\textwidth}
        \includegraphics[width=\textwidth]{25-5}
        \caption{G-N-Iteration25-subgraph5}
        \label{fig: Iteraion25-Subgraph5}
    \end{subfigure}
\end{figure} 


We know the result of the Karate Club (Zachary, 1977) split.
Prove or disprove that the result of split could have been predicted
by the weighted graph of social interactions.  How well does the
mathematical model represent reality?

Generously document your answer with all supporting equations, code,
graphs, arguments, etc.

Useful sources include:

* Original paper

\url{http://aris.ss.uci.edu/~lin/76.pdf}

* Slides

\url{http://www-personal.umich.edu/~ladamic/courses/networks/si614w06/ppt/lecture18.ppt}

\url{http://clair.si.umich.edu/si767/papers/Week03/Community/CommunityDetection.pptx}\\

* Code and data

\url{https://networkx.readthedocs.io/en/stable/examples/graph/karate_club.html}\\

\url{http://nbviewer.ipython.org/url/courses.cit.cornell.edu/info6010/resources/11notes.ipynb}\\

\url{http://stackoverflow.com/questions/9471906/what-are-the-differences-between-community-detection-algorithms-in-igraph/9478989#9478989}\\

\url{http://stackoverflow.com/questions/5822265/are-there-implementations-of-algorithms-for-community-detection-in-graphs}\\

\url{http://konect.uni-koblenz.de/networks/ucidata-zachary}\\

\url{http://vlado.fmf.uni-lj.si/pub/networks/data/ucinet/ucidata.htm#zachary}\\

\url{https://snap.stanford.edu/snappy/doc/reference/CommunityGirvanNewman.html}\\

\url{http://igraph.org/python/doc/igraph-pysrc.html#Graph.community_edge_betweenness}\\

    

\textbf{Solution 1:}\\

Karate club contains 34 club members with Mr. Hi as the club instructor and  John A. as the club president. After the disagreement between Mr. Hi and John A., the club split into 2 different groups, leaving Mr. Hi as the head of one group and John A. as the head of the second group.\\

In order to prove if split could be predicted based on the weighted graph of social interactions, let us consider centrality of the graph. According to Girvan and Newman, The edge betweeness centrality is the shortest path that goes through a graph or network. The centrality gives the measure of important nodes in a graph. If the Zarachy Karate club graph shows two or more important central nodes, this could indicate the possibility of the split.\\

 Girvan and Newman Algorithm states as follows:\\

1. Calculate the betweenness for all edges in the network.

2. Remove the edge with the highest betweenness.

3. Recalculate betweennesses for all edges affected by the removal.

4. Repeat from step 2 until no edges remain.\\


I implemented the Girvan-Newman algorithm iteratively to see if I could possibly get the split showing different subgraphs . Specifically, the John A and Mr. Hi subgraphs. \textit{function GirvinNewmanAlg} in listing 1 shows how I implemented Girvan-Newman algorithim by iteratively removing the edge with the highest betweeness until the graph split into 2 subgraphs. The result of the first split is seen in Figure 3(b) G-N-Iteration12-subgraph2\\

Comparing my result with Zachary's model: From my implementation and graph in Figure 3(b) G-N-Iteration12-subgraph2 I missed node 3 and Zachary missed node 9. This shows that Zachary Karate club split is possible with some level of accuracy.


 \end{homeworkProblem}
%----------------------------------------------------------------------------------------
% PROBLEM 2
%----------------------------------------------------------------------------------------
\begin{homeworkProblem}

We know the group split in two different groups.  Suppose the
disagreements in the group were more nuanced. what would the clubs
look like if they split into groups of 3, 4, and 5?

\textbf{Solution 2:}\\

Since I already predicted that Zachary karate club can be split into 2 by using Girvan and Newman's algorithm, \textit{function GirvinNewmanAlg} in listing 1, I applied the same method to see if the graph could be split into 3, 4, and 5 different subgraphs.

After iteratively applying Girvan and Newman Algorithm, I was able to get the different subgraphs.\\

The graph after 3 split is seen in Figure 4(c) G-N-Iteration15-subgraph3

The graph after 4 split is seen in Figure 4(g) G-N-Iteration19-subgraph4

The graph after 5 split is seen in Figure 4(f) G-N-Iteration25-subgraph5


\end{homeworkProblem}

\nocite{*}
\bibliographystyle{plain}
\bibliography{A5Ref}
\end{document}