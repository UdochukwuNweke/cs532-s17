%%%%%%%%%%%%%%%%%%%%%%%%%%%%%%%%%%%%%%%%%
% Programming/Coding Assignment
% LaTeX Template
%
% This template has been downloaded from:
% http://www.latextemplates.com
%
% Original author:
% Ted Pavlic (http://www.tedpavlic.com)
%
% Note:
% The \lipsum[#] commands throughout this template generate dummy text
% to fill the template out. These commands should all be removed when 
% writing assignment content.
%
% This template uses a Perl script as an example snippet of code, most other
% languages are also usable. Configure them in the "CODE INCLUSION 
% CONFIGURATION" section.
%
%%%%%%%%%%%%%%%%%%%%%%%%%%%%%%%%%%%%%%%%%

%----------------------------------------------------------------------------------------
%	PACKAGES AND OTHER DOCUMENT CONFIGURATIONS
%----------------------------------------------------------------------------------------

\documentclass{article}
\usepackage{fancyhdr} % Required for custom headers
\usepackage{lastpage} % Required to determine the last page for the footer
\usepackage{extramarks} % Required for headers and footers
\usepackage[usenames,dvipsnames]{color} % Required for custom colors
\usepackage{graphicx} % Required to insert images
\usepackage{caption}
\usepackage{listings} % Required for insertion of code
\usepackage{courier} % Required for the courier font
\usepackage{lipsum} % Used for inserting dummy 'Lorem ipsum' text into the template
\usepackage[colorlinks=true,linkcolor=black,anchorcolor=black,citecolor=black,menucolor=black,runcolor=black,urlcolor=black,bookmarks=true]{hyperref}
\usepackage[table,svgnames]{xcolor}
\usepackage{tabularx}
\usepackage{booktabs}
\usepackage{natbib}
\usepackage{hyperref}
\usepackage{natbib}
\usepackage{underscore}
\usepackage{subfigure}

% Margins
\topmargin=-0.45in
\evensidemargin=0in
\oddsidemargin=0in
\textwidth=6.5in
\textheight=9.0in
\headsep=0.25in

\linespread{1.1} % Line spacing

% Set up the header and footer
\pagestyle{fancy}
\lhead{\hmwkAuthorName} % Top left header
\chead{\hmwkClass\ (\hmwkClassInstructor\ \hmwkClassTime): \hmwkTitle} % Top center head
\rhead{\firstxmark} % Top right header
\lfoot{\lastxmark} % Bottom left footer
\cfoot{} % Bottom center footer
\rfoot{Page\ \thepage\ of\ \protect\pageref{LastPage}} % Bottom right footer
\renewcommand\headrulewidth{0.4pt} % Size of the header rule
\renewcommand\footrulewidth{0.4pt} % Size of the footer rule

\setlength\parindent{0pt} % Removes all indentation from paragraphs

%----------------------------------------------------------------------------------------
%	CODE INCLUSION CONFIGURATION
%----------------------------------------------------------------------------------------

\definecolor{MyDarkGreen}{rgb}{0.0,0.4,0.0} % This is the color used for comments
\lstloadlanguages{Perl} % Load Perl syntax for listings, for a list of other languages supported see: ftp://ftp.tex.ac.uk/tex-archive/macros/latex/contrib/listings/listings.pdf
\lstset{language=Perl, % Use Perl in this example
        frame=single, % Single frame around code
        basicstyle=\small\ttfamily, % Use small true type font
        keywordstyle=[1]\color{Blue}\bf, % Perl functions bold and blue
        keywordstyle=[2]\color{Purple}, % Perl function arguments purple
        keywordstyle=[3]\color{Blue}\underbar, % Custom functions underlined and blue
        identifierstyle=, % Nothing special about identifiers                                         
        commentstyle=\usefont{T1}{pcr}{m}{sl}\color{MyDarkGreen}\small, % Comments small dark green courier font
        stringstyle=\color{Purple}, % Strings are purple
        showstringspaces=false, % Don't put marks in string spaces
        tabsize=5, % 5 spaces per tab
        %
        % Put standard Perl functions not included in the default language here
        morekeywords={rand},
        %
        % Put Perl function parameters here
        morekeywords=[2]{on, off, interp},
        %
        % Put user defined functions here
        morekeywords=[3]{test},
       	%
        morecomment=[l][\color{Blue}]{...}, % Line continuation (...) like blue comment
        numbers=left, % Line numbers on left
        firstnumber=1, % Line numbers start with line 1
        numberstyle=\tiny\color{Blue}, % Line numbers are blue and small
        stepnumber=5 % Line numbers go in steps of 5
}

% Creates a new command to include a perl script, the first parameter is the filename of the script (without .pl), the second parameter is the caption




%----------------------------------------------------------------------------------------
%	DOCUMENT STRUCTURE COMMANDS
%	Skip this unless you know what you're doing
%----------------------------------------------------------------------------------------

% Header and footer for when a page split occurs within a problem environment
\newcommand{\enterProblemHeader}[1]{
\nobreak\extramarks{#1}{#1 continued on next page\ldots}\nobreak
\nobreak\extramarks{#1 (continued)}{#1 continued on next page\ldots}\nobreak
}

% Header and footer for when a page split occurs between problem environments
\newcommand{\exitProblemHeader}[1]{
\nobreak\extramarks{#1 (continued)}{#1 continued on next page\ldots}\nobreak
\nobreak\extramarks{#1}{}\nobreak
}

\setcounter{secnumdepth}{0} % Removes default section numbers
\newcounter{homeworkProblemCounter} % Creates a counter to keep track of the number of problems

\newcommand{\homeworkProblemName}{}
\newenvironment{homeworkProblem}[1][Problem \arabic{homeworkProblemCounter}]{ % Makes a new environment called homeworkProblem which takes 1 argument (custom name) but the default is "Problem #"
\stepcounter{homeworkProblemCounter} % Increase counter for number of problems
\renewcommand{\homeworkProblemName}{#1} % Assign \homeworkProblemName the name of the problem
\section{\homeworkProblemName} % Make a section in the document with the custom problem count
\enterProblemHeader{\homeworkProblemName} % Header and footer within the environment
}{
\exitProblemHeader{\homeworkProblemName} % Header and footer after the environment
}

\newcommand{\problemAnswer}[1]{ % Defines the problem answer command with the content as the only argument
\noindent\framebox[\columnwidth][c]{\begin{minipage}{0.98\columnwidth}#1\end{minipage}} % Makes the box around the problem answer and puts the content inside
}

\newcommand{\homeworkSectionName}{}
\newenvironment{homeworkSection}[1]{ % New environment for sections within homework problems, takes 1 argument - the name of the section
\renewcommand{\homeworkSectionName}{#1} % Assign \homeworkSectionName to the name of the section from the environment argument
\subsection{\homeworkSectionName} % Make a subsection with the custom name of the subsection
\enterProblemHeader{\homeworkProblemName\ [\homeworkSectionName]} % Header and footer within the environment
}{
\enterProblemHeader{\homeworkProblemName} % Header and footer after the environment
}

%----------------------------------------------------------------------------------------
%	NAME AND CLASS SECTION
%----------------------------------------------------------------------------------------

\newcommand{\hmwkTitle}{A10} % Assignment title
\newcommand{\hmwkDueDate}{Monday,\ May\ 1,\ 2017} % Due date
\newcommand{\hmwkClass}{\ INTRO. TO WEB SCIENCE:\ CS 532} % Course/class
\newcommand{\hmwkClassTime}{} % Class/lecture time
\newcommand{\hmwkClassInstructor}{Dr. Nelson} % Teacher/lecturer
\newcommand{\hmwkAuthorName}{Udochukwu Nweke} % Your name

%----------------------------------------------------------------------------------------
%	TITLE PAGE
%----------------------------------------------------------------------------------------

\title{
\vspace{2in}
\textmd{\textbf{\hmwkClass:\ \hmwkTitle}}\\
\normalsize\vspace{0.1in}\small{Due\ on\ \hmwkDueDate}\\
\vspace{0.1in}\large{\textit{\hmwkClassInstructor\ \hmwkClassTime}}
\vspace{3in}
}

\author{\textbf{\hmwkAuthorName}}
\date{} % Insert date here if you want it to appear below your name

%----------------------------------------------------------------------------------------

\begin{document}

\maketitle

%----------------------------------------------------------------------------------------
%	TABLE OF CONTENTS
%----------------------------------------------------------------------------------------

%\setcounter{tocdepth}{1} % Uncomment this line if you don't want subsections listed in the ToC

\newpage
\tableofcontents
%\newpage

%----------------------------------------------------------------------------------------
%	PROBLEM 1
%----------------------------------------------------------------------------------------

% To have just one problem per page, simply put a \clearpage after each problem

\begin{homeworkProblem}

\lstinputlisting[caption= Computing K Nearest Neighbours Code, language=python]{P1.py}

\lstinputlisting[caption=Pci Code, language=python]{numpredict.py}

Support your answer: include all relevant discussion, assumptions,
examples, etc.\\

  Using the data from A8:\\

Consider each row in the blog-term matrix as a 1000 dimension vector, 
corresponding to a blog.\\  

- From chapter 8, replace numpredict.euclidean() with cosine as the 
distance metric.  In other words, you'll be computing the cosine between
vectors of 1000 dimensions.\\  

- Use knnestimate() to compute the nearest neighbors for both:

\url{http://f-measure.blogspot.com/}\\
\url{http://ws-dl.blogspot.com/}

for k={1,2,5,10,20}.\\



\textbf{Solution 1:}\\

1. I extracted vectors of  \url{http://f-measure.blogspot.com/} and \url{http://ws-dl.blogspot.com/} from blog-term matrix and considered each row as a 1000 dimension vector. \textit{blogMatrix} was created in Assignment 8. This was achieved by using \textit{do_readfile()} and  \textit{createVectorData()} in listing 1.\\

2. In order to computer cosine similarity between vectors of 1000 dimension (in blogMatix.txt) and  \url{http://f-measure.blogspot.com/} and \url{http://ws-dl.blogspot.com/} blog, I modified Pci \textit{numpredic.py} code by adding line 40-65 in listing 2 which replaced euclidean distance function with cosine similarity function.\\

3. In order to compute K nearest neigbbors for \url{http://f-measure.blogspot.com/} and \url{http://ws-dl.blogspot.com/}, I used \textit{knnestimate()} in listing 1 to computer  for k={1,2,5,10,20} nearest neighbors respectively. The result for \url{http://f-measure.blogspot.com/} K Neighbors  is seen from Table \ref{tab:K=1 Neighbors} to Table \ref{tab:K=20 Neighbors}.  The result for \url{http://ws-dl.blogspot.com/} K Neighbors  is seen from Table \ref{tab:K=1 Neighbor wsdl} to Table \ref{tab:K=20 Neighbors wsdl}.



\begin{table}[h!]
 \centering
 \caption{K= 1 Nearest Neighbors f-measure}
  \label{tab:K=1 Neighbors}
    \begin{tabular}{ |l| l | }
     \hline
 K=1 & Nearest Neighbors\\
 \hline
 1&the fast break of champions \\
 \hline
 \end{tabular}
 \end{table}


 \begin{table}[h!]
 \centering
 \caption{K= 2 Nearest Neighbors f-measure}
  \label{tab:K=2 Neighbors}
    \begin{tabular}{ |l| l | }
     \hline
 K=2 & Nearest Neighbors\\
 \hline
 1&the fast break of champions \\
 \hline
 2&The Jeopardy of Contentment\\
 \hline

 \end{tabular}
 \end{table}


\begin{table}[h!]
 \centering
 \caption{K= 5 Nearest Neighbors f-measure}
  \label{tab:K=5 Neighbors}
    \begin{tabular}{ |l| l | }
     \hline
 K=5 & Nearest Neighbors\\
 \hline
 1&the fast break of champions \\
 \hline
 2&The Jeopardy of Contentment\\
 \hline
 3&The Girl at the Rock Show\\
 \hline
 4&I/LOVE/TOTAL/DESTRUCTION\\
 \hline
 5&Cherry Area\\
 \hline

\end{tabular}
 \end{table}


\begin{table}[h!]
 \centering
 \caption{K= 10 Nearest Neighbors f-measure}
  \label{tab:K=10 Neighbors}
    \begin{tabular}{ |l| l | }
     \hline
 K=10 & Nearest Neighbors\\
 \hline
 1&the fast break of champions \\
 \hline
 2&The Jeopardy of Contentment\\
 \hline
 3&The Girl at the Rock Show\\
 \hline
 4&I/LOVE/TOTAL/DESTRUCTION\\
 \hline
 5&Cherry Area\\
 \hline
 6&SunStock Music\\
 \hline
 7&In the Frame Film Reviews\\
 \hline
 8&CardrossManiac2\\
 \hline
 9&Encore\\
 \hline
 10&The Stark Online\\
 \hline

 \end{tabular}
\end{table}


\begin{table}[h!]
 \centering
 \caption{K= 20 Nearest Neighbors f-measure}
  \label{tab:K=20 Neighbors}
    \begin{tabular}{ |l| l | }
     \hline
 K=20 & Nearest Neighbors\\
 \hline
 1&the fast break of champions \\
 \hline
 2&The Jeopardy of Contentment\\
 \hline
 3&The Girl at the Rock Show\\
 \hline
 4&I/LOVE/TOTAL/DESTRUCTION\\
 \hline
 5&Cherry Area\\
 \hline
 6&SunStock Music\\
 \hline
 7&In the Frame Film Reviews\\
 \hline
 8&CardrossManiac2\\
 \hline
 9&Encore\\
 \hline
 10&The Stark Online\\
 \hline
 11&Steel City Rust\\
 \hline
 12&Diagnosis: No Radio\\
 \hline
 13&www.doginasweater.com Live Show Review Archive\\
 \hline
 14&Some Call It Noise....\\
 \hline
 15&15 Cuz Music Rocks\\
 \hline
 16&She May Be Naked\\
 \hline
 17&GLI Press\\
 \hline
 18&Morgan's Blog\\
 \hline
 19&the fast break of champions\\
 \hline
 20&Pithy Title Here\\
 \hline

 \end{tabular}
\end{table}



\begin{table}[h!]
 \centering
 \caption{K= 1 Nearest Neighbor WSDL}
  \label{tab:K=1 Neighbor wsdl}
    \begin{tabular}{ |l| l | }
     \hline
 K=1 & Nearest Neighbors\\
 \hline
 1&tmacthemost \\
 \hline
\end{tabular}
\end{table}



\begin{table}[h!]
 \centering
 \caption{K= 2 Nearest Neighbors WSDL}
  \label{tab:K=2 Neighbors wsdl}
    \begin{tabular}{ |l| l | }
     \hline
 K=2 & Nearest Neighbors\\
 \hline
 1&tmacthemost \\
 \hline
 2&the traveling neighborhood\\
 \hline

\end{tabular}
\end{table}



\begin{table}[h!]
 \centering
 \caption{K= 5 Nearest Neighbors WSDL}
  \label{tab:K=5 Neighbors wsdl}
    \begin{tabular}{ |l| l | }
     \hline
 K=5 & Nearest Neighbors\\
 \hline
 1&tmacthemost \\
 \hline
 2&the traveling neighborhood\\
 \hline
 3&MarkFisher's-MusicReview\\
 \hline
 4&Bonjour Girl\\
 \hline
 5&Avidd Wallows' Blog\\
 \hline

\end{tabular}
\end{table}

\begin{table}[h!]
 \centering
 \caption{K= 10 Nearest Neighbors WSDL}
  \label{tab:K=10 Neighbors wsdl}
    \begin{tabular}{ |l| l | }
     \hline
 K=10 & Nearest Neighbors\\
 \hline
 1&tmacthemost \\
 \hline
 2&the traveling neighborhood\\
 \hline
 3&MarkFisher's-MusicReview\\
 \hline
 4&Bonjour Girl\\
 \hline
 5&Avidd Wallows' Blog\\
 \hline
 6&STATUS\\
 \hline
 7&Cherry Area\\
 \hline
 8&A2 MEDIA COURSEWORK JOINT BLOG\\
 \hline
 9&The Stark Online\\
 \hline
 10&Pithy Title Here\\
 \hline

\end{tabular}
\end{table}


\begin{table}[h!]
 \centering
 \caption{K= 20 Nearest Neighbors WSDL}
  \label{tab:K=20 Neighbors wsdl}
    \begin{tabular}{ |l| l | }
     \hline
 K=20 & Nearest Neighbors\\
 \hline
 1&tmacthemost \\
 \hline
 2&the traveling neighborhood\\
 \hline
 3&MarkFisher's-MusicReview\\
 \hline
 4&Bonjour Girl\\
 \hline
 5&Avidd Wallows' Blog\\
 \hline
 6&STATUS\\
 \hline
 7&Cherry Area\\
 \hline
 8&A2 MEDIA COURSEWORK JOINT BLOG\\
 \hline
 9&The Stark Online\\
 \hline
 10&Pithy Title Here\\
 \hline
 11&Floorshime Zipper Boots\\
 \hline
 12&juanbook\\
 \hline
 13&Punk Rock Teaching\\
 \hline
 14&Chantelle Swain A2 Media Studies\\
 \hline
 15&Myopiamuse\\
 \hline
 16&She May Be Naked\\
 \hline
 17&Kid F\\
 \hline
 18&the fast break of champions\\
 \hline
 19&tDiagnosis: No Radio\\
 \hline
 20&20 Mile In Mine\\
 \hline

 \end{tabular}
\end{table}



\end{homeworkProblem}
%----------------------------------------------------------------------------------------
% PROBLEM 2
%----------------------------------------------------------------------------------------






%----------------------------------------------------------------------------------------
% PROBLEM 3
%----------------------------------------------------------------------------------------









\nocite{*}\clearpage
\bibliographystyle{plain}
\bibliography{A10Ref}


\end{document}
