%%%%%%%%%%%%%%%%%%%%%%%%%%%%%%%%%%%%%%%%%
% Programming/Coding Assignment
% LaTeX Template
%
% This template has been downloaded from:
% http://www.latextemplates.com
%
% Original author:
% Ted Pavlic (http://www.tedpavlic.com)
%
% Note:
% The \lipsum[#] commands throughout this template generate dummy text
% to fill the template out. These commands should all be removed when 
% writing assignment content.
%
% This template uses a Perl script as an example snippet of code, most other
% languages are also usable. Configure them in the "CODE INCLUSION 
% CONFIGURATION" section.
%
%%%%%%%%%%%%%%%%%%%%%%%%%%%%%%%%%%%%%%%%%

%----------------------------------------------------------------------------------------
%	PACKAGES AND OTHER DOCUMENT CONFIGURATIONS
%----------------------------------------------------------------------------------------

\documentclass{article}

\usepackage{fancyhdr} % Required for custom headers
\usepackage{lastpage} % Required to determine the last page for the footer
\usepackage{extramarks} % Required for headers and footers
\usepackage[usenames,dvipsnames]{color} % Required for custom colors
\usepackage{graphicx} % Required to insert images
\usepackage{listings} % Required for insertion of code
\usepackage{courier} % Required for the courier font
\usepackage{lipsum} % Used for inserting dummy 'Lorem ipsum' text into the template
\usepackage[colorlinks=true,linkcolor=black,anchorcolor=black,citecolor=black,menucolor=black,runcolor=black,urlcolor=black,bookmarks=true]{hyperref}
\usepackage{breakurl}
\usepackage{hyperref}


% Margins
\topmargin=-0.45in
\evensidemargin=0in
\oddsidemargin=0in
\textwidth=6.5in
\textheight=9.0in
\headsep=0.25in

\linespread{1.1} % Line spacing

% Set up the header and footer
\pagestyle{fancy}
\lhead{\hmwkAuthorName} % Top left header
\chead{\hmwkClass\ (\hmwkClassInstructor\ \hmwkClassTime): \hmwkTitle} % Top center head
\rhead{\firstxmark} % Top right header
\lfoot{\lastxmark} % Bottom left footer
\cfoot{} % Bottom center footer
\rfoot{Page\ \thepage\ of\ \protect\pageref{LastPage}} % Bottom right footer
\renewcommand\headrulewidth{0.4pt} % Size of the header rule
\renewcommand\footrulewidth{0.4pt} % Size of the footer rule

\setlength\parindent{0pt} % Removes all indentation from paragraphs

%----------------------------------------------------------------------------------------
%	CODE INCLUSION CONFIGURATION
%----------------------------------------------------------------------------------------

\definecolor{MyDarkGreen}{rgb}{0.0,0.4,0.0} % This is the color used for comments
\lstloadlanguages{Perl} % Load Perl syntax for listings, for a list of other languages supported see: ftp://ftp.tex.ac.uk/tex-archive/macros/latex/contrib/listings/listings.pdf
\lstset{language=Perl, % Use Perl in this example
        frame=single, % Single frame around code
        basicstyle=\small\ttfamily, % Use small true type font
        keywordstyle=[1]\color{Blue}\bf, % Perl functions bold and blue
        keywordstyle=[2]\color{Purple}, % Perl function arguments purple
        keywordstyle=[3]\color{Blue}\underbar, % Custom functions underlined and blue
        identifierstyle=, % Nothing special about identifiers                                         
        commentstyle=\usefont{T1}{pcr}{m}{sl}\color{MyDarkGreen}\small, % Comments small dark green courier font
        stringstyle=\color{Purple}, % Strings are purple
        showstringspaces=false, % Don't put marks in string spaces
        tabsize=5, % 5 spaces per tab
        %
        % Put standard Perl functions not included in the default language here
        morekeywords={rand},
        %
        % Put Perl function parameters here
        morekeywords=[2]{on, off, interp},
        %
        % Put user defined functions here
        morekeywords=[3]{test},
       	%
        morecomment=[l][\color{Blue}]{...}, % Line continuation (...) like blue comment
        numbers=left, % Line numbers on left
        firstnumber=1, % Line numbers start with line 1
        numberstyle=\tiny\color{Blue}, % Line numbers are blue and small
        stepnumber=5 % Line numbers go in steps of 5
}

% Creates a new command to include a perl script, the first parameter is the filename of the script (without .pl), the second parameter is the caption
\newcommand{\perlscript}[2]{
\begin{itemize}
\item[]\lstinputlisting[caption=#2,label=#1]{#1.pl}
\end{itemize}
}

%----------------------------------------------------------------------------------------
%	DOCUMENT STRUCTURE COMMANDS
%	Skip this unless you know what you're doing
%----------------------------------------------------------------------------------------

% Header and footer for when a page split occurs within a problem environment
\newcommand{\enterProblemHeader}[1]{
\nobreak\extramarks{#1}{#1 continued on next page\ldots}\nobreak
\nobreak\extramarks{#1 (continued)}{#1 continued on next page\ldots}\nobreak
}

% Header and footer for when a page split occurs between problem environments
\newcommand{\exitProblemHeader}[1]{
\nobreak\extramarks{#1 (continued)}{#1 continued on next page\ldots}\nobreak
\nobreak\extramarks{#1}{}\nobreak
}

\setcounter{secnumdepth}{0} % Removes default section numbers
\newcounter{homeworkProblemCounter} % Creates a counter to keep track of the number of problems

\newcommand{\homeworkProblemName}{}
\newenvironment{homeworkProblem}[1][Problem \arabic{homeworkProblemCounter}]{ % Makes a new environment called homeworkProblem which takes 1 argument (custom name) but the default is "Problem #"
\stepcounter{homeworkProblemCounter} % Increase counter for number of problems
\renewcommand{\homeworkProblemName}{#1} % Assign \homeworkProblemName the name of the problem
\section{\homeworkProblemName} % Make a section in the document with the custom problem count
\enterProblemHeader{\homeworkProblemName} % Header and footer within the environment
}{
\exitProblemHeader{\homeworkProblemName} % Header and footer after the environment
}

\newcommand{\problemAnswer}[1]{ % Defines the problem answer command with the content as the only argument
\noindent\framebox[\columnwidth][c]{\begin{minipage}{0.98\columnwidth}#1\end{minipage}} % Makes the box around the problem answer and puts the content inside
}

\newcommand{\homeworkSectionName}{}
\newenvironment{homeworkSection}[1]{ % New environment for sections within homework problems, takes 1 argument - the name of the section
\renewcommand{\homeworkSectionName}{#1} % Assign \homeworkSectionName to the name of the section from the environment argument
\subsection{\homeworkSectionName} % Make a subsection with the custom name of the subsection
\enterProblemHeader{\homeworkProblemName\ [\homeworkSectionName]} % Header and footer within the environment
}{
\enterProblemHeader{\homeworkProblemName} % Header and footer after the environment
}

%----------------------------------------------------------------------------------------
%	NAME AND CLASS SECTION
%----------------------------------------------------------------------------------------

\newcommand{\hmwkTitle}{Assignment\ \#2} % Assignment title
\newcommand{\hmwkDueDate}{Thursday,\ February\ 9,\ 2017} % Due date
\newcommand{\hmwkClass}{ Introduction to Web Science} % Course/class
\newcommand{\hmwkClassTime}{} % Class/lecture time
\newcommand{\hmwkClassInstructor}{Dr. Nelson} % Teacher/lecturer
\newcommand{\hmwkAuthorName}{Udochukwu Nweke} % Your name

%----------------------------------------------------------------------------------------
%	TITLE PAGE
%----------------------------------------------------------------------------------------

\title{
\vspace{2in}
\textmd{\textbf{\hmwkClass:\ \hmwkTitle}}\\
\normalsize\vspace{0.1in}\small{Due\ on\ \hmwkDueDate}\\
\vspace{0.1in}\large{\textit{\hmwkClassInstructor\ \hmwkClassTime}}
\vspace{3in}
}

\author{\textbf{\hmwkAuthorName}}
\date{} % Insert date here if you want it to appear below your name

%---------------------------------------------------------------------------------------
\begin{document}



\maketitle

%----------------------------------------------------------------------------------------
%	TABLE OF CONTENTS
%----------------------------------------------------------------------------------------

%\setcounter{tocdepth}{1} % Uncomment this line if you don't want subsections listed in the ToC

\newpage
\tableofcontents
\newpage

%----------------------------------------------------------------------------------------
%	PROBLEM 1
%----------------------------------------------------------------------------------------

% To have just one problem per page, simply put a \clearpage after each problem

\begin{homeworkProblem}

Write a python program that extracts 1000 unique links from twitter.


\perlscript{A2_P1}{Code for extracting links from Twitter}

1.  I used Twitter search API to periodically search for tweets with search terms: ``Trump'', ``Obama'', ``Sports'', and ``Music''. I only searched for tweets that were not truncated. If the tweet's truncated flag, was set, I skipped the tweet (Listing 1 line 19 demonstrates how I avoided truncated Tweets).\\

2. For each of these tweets, I extracted links (expanded links to remove t.co) from tweets that have not been truncated, because truncated tweets (which have a self reference) will require another API call with \textit{tweet\_mode=extended} option in other to get a link that does not reference itself.\\


Listing 2 is a snippet of the extracted raw file. The complete file is in \textit{allLinks.txt} file.\\

\perlscript{A2_P1RawFiles}{A snippet of allLinks.txt containing 6 links extracted from tweets}
 3.  I grabbed 5,220 links and removed duplicate links by using a dictionary  to store key value pairs where (url-key and id-value)since dictionaries do not permit duplicates. I  unshortened the urls to get the final links.\\
 
Listing 1, line 52 shows how I removed duplicates in other to get unique urls with \textit{removeDups()}. After removing duplicate links the 5,220 urls dropped to 1,900 urls. Hence, I grabbed only a 1000 url and saved in \textit{1000UniqueUrls.txt} file.
.

\end{homeworkProblem}

%----------------------------------------------------------------------------------------
%	PROBLEM 2
%----------------------------------------------------------------------------------------

\begin{homeworkProblem}
2. Download the Timemaps for each of the target URLs. We'll use the ODU Memento Aggregator, so for example:

URI-R = \url{http://www.cs.odu.edu/}

URI-T = \url{http://memgator.cs.odu.edu/timemap/link/http://www.cs.odu.edu/}

or:

URI-T \url{http://memgator.cs.odu.edu/timemap/json/http://www.cs.odu.edu}

(depending on which format you would prefer to parse)

Create a histogram* of URIs vs. number of mementos (as computed from the TimeMaps). For example, 100 URIs with 0 Memntos, 300 URIs with 1 Memento, 400 URIs with 2 Mementos, etc. The x-axis will have the number of mementos, and the y-axis will have the frequency of occurence.

* = \url{http://en.wikipedia.org/wiki/Histogram}

What's a TimeMap?
See:\url{ http://www.mementoweb.org/Wiki/Histogram/guide/quick-intro/}
And the Week 4 lecture.


\perlscript{A2_P2} {Code Snippet For Problem 2}
1. I downloaded TimeMaps for the 1000 unique URLs by using curl to dereference the concatenation of the ODU Memento Aggregator and each url - Listing 3 (\textit{function downloadTimemap})\\

2. After downloading the TimeMaps, I saved only TimeMaps that existed for URLs (because some URLs did not have TimeMaps) - Listing 3 (\textit{function downloadTimemapsAndSave})\\

3. I wrote a function called \textit{countMementos} which extracted and counted mementos for each link - Listing 3\\


4. I wrote the result into a CSV file as shown in listing 5. The agePlotData.csv file has everything.\\

5. I wrote an R program to plot the histogram.\\

The R code snippet in  listing 4 demonstrates how I used R to plot the histogram\
\perlscript{A2_P2His}{R code For The Histogram}

\perlscript{MementoCount}{creationDate.csv}

\end{homeworkProblem}

\begin{homeworkProblem}
Estimate the age of each of the 1000 URIs using the ``Carbon Date" tool:\\

\url{http://ws-dl.blogspot.com/2016/09/2016-09-20-carbon-dating-web-version-30.html}\\

But it will inevitably crash when everyone tries to use it at the last minute.\\

For URIs that have $>$ 0 Mementos and estimated creation date, create a graph with age (in days) on the x-axis and number of mementos on the y-axis.\\

Not all URIs will have Mementos, not all URIs will have an estimated creation date. Show how many fall into either categories. For example,\\

total URIs: 1000\newline
no mementos: 137\newline
no date estimate: 212\\

\perlscript{A2_P3}{Extract Memento count and Age (CarbonDate)}
\includegraphics{RplotsA2P2}
\includegraphics{RplotsA2P3}

1. In order to generate the estimated age of the 1000 URIs, the docker application is used to call CarbonDate. This was done by \textit{getCreationDateForURI} function - Listing 5, line 6.\\

2. The \textit{getCreationDates} function in Listing 5 line 24 is responsible for writing the creation date for the 1000 URIs. This function writes the Age into \textit{creationDates.csv} file.\\

3. To generate a plot for urls with mementos and creation dates, I created a dictionary which matches urls from both files the \textit{MementoCount.csv} and \textit{CreationDate.csv} files - Listing 5, \textit{matchMementoCountWithCreationDates function}.\\
The result is \textit{agePlotData.csv} files.

\perlscript{AgePlotData}{Extract agePlotData.csv}

\perlscript{A2_P3Plot}{R code matchMementoCreationDates Graph}
Total URIs: 1000\newline\
Number of Mementos: 260\newline\
Number of Date Estimate: 230

\end{homeworkProblem}\

\textbf{References}\\

1.  \url{http://docs.tweepy.org/en/v3.5.0/index.html}\\

2. \url{https://github.com/bear/python-twitter}\\

3. \url{https://dev.twitter.com/rest/public}\\

4.  \url{http://memgator.cs.odu.edu/timemap/link/http://www.cs.odu.edu/}\\

5. \url{http://memgator.cs.odu.edu/timemap/json/http://www.cs.odu.edu/}\\

6. \url{https://en.wikipedia.org/wiki/Histogram}\\

7. \url{http://ws-dl.blogspot.com/2016/09/2016-09-20-carbon-dating-web-version-30.html}

%--------------------------------------------------------------------------------
\end{document}

